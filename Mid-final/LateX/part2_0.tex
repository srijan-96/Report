\chapter{Getting started}
\section{Limitations of the present HTML editor in OpenEdx}
Open edX in itself is a humongous learning management system with unparalleled feature list. It is
among the top MOOCs platforms in the world. As it is the way of the world, there is nothing perfect
here and Open edX is not an exception. All applications/platforms have a scope to enhance its
features and abilities of its components. In this project we worked on enhancing the HTML
component which is used for content creation in courses.\newline\newline
While current HTML editor does the job well but it has got its own mind and behaves with
uncertainty sometimes. Enhancing the editor with internal CSS and ironing out the indentation
issues will boost the user experience of the course creator. Our aim was to develop and integrate a
full-fledged HTML editor like Notepad++ or Sublime for Open edX.\newline\newline
Here are some issues that we have encountered in the HTML editor used by the OpenEdx platform
\begin{enumerate}
  \item \textbf{Indentaion in the current HTML editor :-}\newline\newline
The purpose of code indentation and style is to make the program more readable and
understandable. It saves lots of time while we revisit the code and use it.
Code indentaion in an important part towards increasing the aesthetic of any editor,
as it increases readability and makes understanding of the code easier.\newline\newline
In Open edX however the code indentation feature is not implemented in the raw
HTML editor, which is the CodeMirror v3.21.0. This is mainly because the version
of CodeMirror used in Open edX has been implemented with a minimalistic use case
and has not been updated since October 2015. The problem is that even if we
manually indent our code, once we save our progress and close the raw editor, the
indentation is lost i.e it doesn’t gets saved. 

  \item \textbf{Code Folding in the current HTML editor :-}\newline\newline
Code folding is a feature of some text editors, source code editors, and IDEs that
allows the user to selectively hide and display – "fold" – sections of a currently edited
file as a part of routine edit operations. This allows the user to manage large
amounts of text while viewing only those subsections of the text that are specifically
relevant at any given time.\newline\newline
This is a pretty useful feature for any editor and presents the user with various use
patterns, primarily organizing code or hiding less useful information so one can
focus on more important information. Since the CodeMirror version implemented in
Open edX is very old this feature is not available and thus was added to our List of
possible enhancements.

  \item \textbf{Internal CSS in HTML Code :-}\newline\newline
Cascading Style Sheets (CSS) is a style sheet language used for describing the
presentation of a document written in a markup language like HTML CSS is a
cornerstone technology of the World Wide Web alongside HTML and JavaScript.\newline\newline
The three main types of CSS styles are inline, internal and external. Currently in
Open edX platform in the raw HTML editor only inline or embedded CSS support is
available. Inline styles are CSS styles that are applied directly in the page's HTML.
Because inline styles they are the most specific in the cascade, they can over-ride
things we didn't intend them to. They also negate one of the most powerful aspects
of CSS - the ability to style lots and lots of web pages from one central CSS file to
make future updates and style changes much easier to manage.\newline\newline
If we had to only use inline styles, our documents would quickly become bloated and
very hard to maintain. This is because inline styles must be applied to every element
we want them on. That is why internal CSS is preferred whenever exhaustive use of
CSS is required as opposed to inline CSS. Internal CSS is Open edx raw HTML
editor is not supported as internal CSS must be defined inside the style tag in the
head element of the html body. However the raw HTML editor of Open edX stips
down style tags and the effects are not reflected as wanted. Styles of other
components in the web page also gets changed which we do not want. Thus allowing
internal CSS is an important enhancement criterion which should be addressed.
\end{enumerate}

To understand the limitations better we have provided the screenshots as mentioned
below:
\begin{center}\textbf{[See Figures 1-4 in the “Figures and Screenshots” section.]}\end{center}


\section{Objectives}
Here are the list of enhancements we were aiming for in the html component of Open edX:
\begin{itemize}
  \item Improvise the raw HTML editor to have features like pretty indentation, code folding
etc.
  \item Check if we can add a separate CSS section in raw HTML editor to eliminate inline
CSS
\item Fix the cursor placement issue and some other minor issues.
\end{itemize}

\section{Observations}
Regarding the Open edX paltform and the editors used in the Open Edx one might come up with thefollowing kind of questions
\begin{itemize}
\item Is OpenEdx using a raw editor or visual editor or both ?
\item Did OpenEdx write their own editor ?
\item If yes, then what how is it working ?
\item If no, then from where they have implemented their text editors
\item What kind of editor are they using ?
\item Where do we find the editors source code ?
\item How many editors are they using ?
\item Their method of working ?
\item Their current scenario ?
\item On what basis is this editor working along with the OpenEdx platform ?\newline\newline
\end{itemize}


To all these queries here are some facts that would be helpful:
\begin{enumerate}
\item OpenEdx uses both the visual editor as well as the raw editor.
\item OpenEdx has not written their own editor neither the visual editor nor the raw editor.
\item OpenEdx uses open source text editors for this purpose.
\item The visual editor is tinyMCE editor and the raw editor is CodeMirror Editor.
\item These editors are located in edx source code, in github.
\item CodeMirror is made to work in the custom mode configuration. OpenEdx is using a very old version of CodeMirror.
\item The tinymce editor folder of openedx consists of a plugin called "codemirror-for-tinymce".
\item The good thing about this plugin is that any version of codemirror can be used in the code and it will work perfectly fine.
\item The “codemirror-for-tinymce” plugin is already in use in the OpenEdx repository.
\item We can use the “codemirror-for-tinymce” plugin for tinymce 4.x.x versions.
\item There is an xblock module or simply an xblock called html-module.py
\item This particular xblock module handles the operation of the html editor.
\item This is the module that loads the html editor widget and configures both (tinymce and codemirror) editors to work hand in hand with the help of “codemirror” plugin for tinymce.
\end{enumerate}


\section{Our plan of action}

Depending on the facts that we uncovered regarding the html editors we have come up with
two possible approaches initially, that would fulfill our required objectives.
\begin{enumerate}

\item\textbf{Approach I}\newline Updating the present version of WYSIWYG and raw HTML editors in Open
edX
\item\textbf{Approach II}\newline Integrating other third party open source text editors into OpenEdx platform.
\end{enumerate}

%%%%%%%%%%%%%%%%%%%%%%%%%%%%%%%%%%%%%%%%%%%%%%%%%%%%%%%%%%%%%%%%%%%%%%%%%

\chapter{Approach I - Local intregation}
\section{Updating the present version of Editors present in Open edX}
Our first approach was to download the latest versions of TinyMCE, CodeMirror and
the plugin “codemirror-for-tinymce” from their respective git repositories into our local
Machine.\newline
Next step was to intregate the two editors i.e., CodeMirror and TinyMCE using the plugin and
test it on our local machine, and later integrate it with Open edX using one of the Open edX developing
environment.\newline
The result would be TinyMCE loading the instance of latest version of CodeMirror as its
raw HTML editor.
However this implementation could only solve the indentation and code folding problem,
but still the internal css problem wouldn’t be solved just with this.\newline\newline
For enabling the internal CSS we thought of two possible implementations:
\begin{enumerate}
\item\textbf{Using Iframes}\newline
For this, we were planning to wrap the entire html content inside an iframe
and then add theming to the iframe.
\begin{itemize}
\item Set the width of iframe to 100\%
\item Set the height of iframe equal to the height of content of iframe
\item Set the border of iframe to zero/none
\item Open all hyperlinks within the iframe in new tab
\end{itemize}
Hence the expected result is an invisible iframe holding our HTML content.



\item\textbf{Using JavaScript}\newline
This is a slightly tough and critical approach, which was to apply CSS using
JavaScript. The plan was to separate out CSS part from HTML content and
wrap the entire content in a html div tag with an unique id.
Next we would have to manually select each and every child of the div by
parsing the DOM (Document Object Model) tree and apply styles to it.\newline
\end{enumerate}

We opted for the first i.e the Iframe approach because of the following reasons:
\begin{enumerate}
\item This method allows the course content creator to include third party CSS such as
bootstrap, w3css which will make the course even more beautiful
\item Implementaion is easier
\item Wrapping data inside the iframes containerizes the data ensuring no mixing of
CSS between two HTML blocks in LMS.
\item The overhead of the second approach will be very large, since parsing
the DOM tree would take considerable amount of time and effort.
\end{enumerate}


\section{Implementation of our Local intregation}
In the first phase of implementing our approach of updating the present editors, we
first downloaded both the editors onto our local machine and then integrated the latest
version of CodeMirror and TinyMCE in our local machine. Later we integrated it with
Django framework since Open edX uses Django framework in the back end as their Web
Application Framework. This step was done to give us a better insight into the interaction of
the present editors in Open edX with Django. The steps of our implementation are
mentioned in brief below:
\begin{enumerate}
\item Firstly, we had downloaded the latest version of CodeMirror and TinyMCE from their
github repositores and then tested them individually to check whether all the desired features
are working or not.
\item Secondly, we had downloaded the latest version of the “codemirror-for-tinymce” plugin
from its git repo.
\item Next we had replaced the default raw HTML editor of TinyMCE with our instance of the
latest version of CodeMirror. This was done by configuring the \textbf{init-tinymce.js} file to use
CodeMirror as an external plugin and thus load it as its default raw HTML editor.
\item Later, for enabling the internal CSS we had encapsulated the entire html content of the
editor into an iframe.
\item We have then set the width of the iframe to 100 percent and set the height of the iframe
equal to the height of the html content.
\item  We had made sure that the borders of iframe was set to zero to make it invisible so that
the user experience is better and smooth.
\item After completing our stand alone integration we merged it with a Django project to run
our editor as an app in the Django server. This intregation was quite useful as Open edX
itself uses Django framework in its backend and it helped us in understanding about the
Django back end of Open edX.
\end{enumerate}


\section{Summary of our local intregation}
After completing our local integration, firstly we had tested our local instances of
TinyMCE and CodeMirror on the Django server. A brief summary of our observations are
listed below.
\begin{enumerate}
\item Most of the HTML tags and their atrributes are working fine. Some of the exceptions
being script tag , audio tag etc.
\item Almost every CSS style and attributes work fine. The exceptions are those style features
which by default require a script tag to be implemented.
\item CSS animations can be implemented and other third party HTML-and CSS-based design
templates such as Bootstrap can be linked and used without any error.
\item We prepared a detailed Test Report which can be found here:\newline[\url{https://docs.google.com/spreadsheets/d/1AJQWfAG2NIjpxFZt2fnLPlKdhP2-q35KZ0DOmdb-F0w/edit#gid=0}]
\item  Source code is available in our github repository here:\newline[\url{https://github.com/ashutoshbsathe/codemirror-in-tinymce}]
\end{enumerate}


\section{Issues faced and solutions}
\begin{enumerate}

\item\textbf{AJAX:}\newline\newline
We had no idea about how to pass data between RESTful APIs at first. Then later we
found out that this can be done via a request called as AJAX which basically passes
the data in JSON format from one API to another API. And hence we used the same.
On a button click in our html, we would send the AJAX query from webpage to
Django backend. While this seems easy at first, it is a bit tricky when we are using
only a single text area and not entire form element. So the main problem was our
csrf\_token was not being set correctly since we were not using a form. So we
decided to completely skip the csrf authentication by adding @csrf\_exempt
decorator to our view

\item\textbf{Stripping of style tags:}\newline\newline
So tinyMCE does not allow you to directly add CSS in your html. It will constantly
filter out your own \verb=<style>= and \verb=<link>= tags out of it’s display content. The only way
to get around this is by setting valid\_children, valid\_elements and
extended\_valid\_elements properly in TinyMCE config.

\item\textbf{Indentation in the “CodeMirror with TinyMCE setup” :}\newline\newline
Now because the tinymce keeps filtering out content, there’s no actual way to show a
nice indented code in CodeMirror. For this we found a script that will indent our
html code nicely and we called it before setting the content in CodeMirror. One thing
to note here is that whatever indentation that was done in CodeMirror was a bit of a
pseudo Indentation and not true indentation as the course creator wants. Sure, it was
nicely indented, but if somewhere content creator specifically wanted a specific tag
on multiple line such as
\begin{verbatim}
<a href=”https://www.google.com”>
A SearchEngine
</a>
\end{verbatim}
It won’t be rendered in that way because the script treats anchor tag as a single-line
tag and it will be rendered as follows:
\begin{verbatim}
<a href=”https://www.google.com”>ASearchEngine</a>
\end{verbatim}
Sadly enough, there is no way for tinyMCE to shut down it’s content filtering and
hence this issue remains unsolved.


\item\textbf{Unnecessary modification of html code:}\newline\newline
This issue specifically occurs when you try to edit something that was added in raw
mode. For example, let’s hope you add a nice CSS animation and then return back to
tinyMCE for saving the content, while returning back, you accidently clicked in the
div of your animation and you pressed enter. Then tinymce will try to split your div
in half breaking it where you pressed enter, and will also add some bad html code
which you definitely don’t want. Again this issue is unavoidable because of content
filtering algorithms used in tinyMCE.
\end{enumerate}

%%%%%%%%%%%%%%%%%%%%%%%%%%%%%%%%%%%%%%%%%%%%%%%5
