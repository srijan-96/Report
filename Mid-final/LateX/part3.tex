\chapter{Research about Approach II}
\section{Integration Of The other third party Editors}
Just like CodeMirror and TinyMCE there are other third party open source text editors
which meet all the prerequisites of our enhancement critera. Some of them are Notepad++,
Quill, Brackets, ContentTools, Grapejs,etc. As an alternative to our first approach we
thought of another possibility of reaching our objective. Just like how the OpenEdx has
merged two text editors for their HTML editor, similarly even we can try to integrate
WYSWYG editor using the same methodology. In this approach we develop an entire new
editor itself. Moreover since all these editors are open source their github repositories is
easily accessible. So we also thought of trying to import the logic they used in their source
code to add features such as code indentation ,code folding etc. We put this approach as our
backup for our approach1 because, approach2 is a bit hacky and complicated to implement
compared to approach1. 
\section{What kind of editors do we want?}  
Before integrating here are some of the questions that we asked ourselves. \newline
\begin{itemize}
\item What kind of editor do we want...??
\item What features does that editor should have..??
\item How does these editors help with our motive..?? \newline 
\end{itemize}
So here are few points that helped us clear our doubts: \newline
\begin{itemize}
\item We need an editor that makes our work easy, simple, and has got much better working
properties as compared to the editor that is currently being used in the OpenEdx platform.
\item The editor should have inbuilt features(code folding, pretty indentation,languages,etc) that
could just wipe out the issues that are being faced by the openEdx html editor.
\item Since our motive is to add features such as indendation , cold folding, internal css,etc, so
considering an editor that has these features inbuilt, and integrating it by making some
required changes might help us in achieving our motive, and would even resolve in
eliminating the issues faced by the current HTML editor that is being used by the OpenEdx
platform. \newline
\end{itemize}

After a bit of research work what we found out is some pretty good and well built editors
that support html language. All these editors are open source and all their source code is
available in their respective git repositories. The list of editors that we had researched on
are as follows: \newline
\begin{itemize}
\item Notepad++
\item Brackets
\item ContentTools
\end{itemize}
The detailed explanation about each editor follows.
\section{Notepad++}
\subsection{About}
Notepad++ is a free (as in “free speech” and also as in “free beer”) open source code editor
and Notepad replacement that supports several languages. Running in the operating system
its use is governed by the GPL license. \newline\newline
As of now the current version of the Notepad++ available in the market is v7.5.6. \newline
\subsection{Features} 
\begin{enumerate}
\item Syntax Highlightning
\item Syntax Folding
\item User Desfined Syntax Highligting and Syntax Folding. (images (n++1,2,3,4))\item PCRE (Perl Compatible Regular Expreession) Search and Replace.
\item GUI entirely customizable.
\item Auto completion: Word completion, Function completion and Function parameter hint.
\item Multi document (Tab interface).
\item Multi view
\item WYSIWYG (printing).
\item Zoom in and Zoom out.
\item Multi language environment supported.
\item Macro recording and playback.
\item Bookmark.  \newline
\end{enumerate}
\subsection{Observations Made} 
Here are a list of things that we have found about notepad++ : 
\begin{itemize}
\item Scintilla is the main component of Notepad++ which is very powerful.
\item This editor is written in c++
\item All the syntax styling part is done using the Scintilla component.
\item Scintilla edits and even is used for debugging the source code of the Notepad++
\item Some files where the required changes are to be made and importing those files so they can
be used in building an editor for OpenEdx platform. \newline
\end{itemize}  

\subsection{Idea Of Implementation }
In a similar way in which the Open edX platform uses an xmodule(html\_module.py) that
handles the operation of its current html editor, similarly, using the same working principle
we can replace our editor and load the html editor and configure it for proper working.  
\section{Adobe Brackets}
\subsection{About}
With focused visual tools and preprocessor support, Brackets is a modern text editor that
makes it easy to design in the browser. It's crafted from the ground up for web designers and
front-end developers. \newline 
Brackets is an open source editor written in HTML, CSS, and JavaScript with a primary
focus on web development. It was created by Adobe Systems, licensed under the MIT
License, and is currently maintained on GitHub by Adobe and other open-sourced
developers. Brackets is available for cross-platform download on Mac, Windows, and is
compatible with most linux distros. The main purpose of brackets is its live html, css and js
editing functionality.
\subsection{Features}
\begin{enumerate}
\item Quick Edit
\item Quick Docs
\item Live element debugging
\item Live preview
\item Inline Editors
\item Split View
\item Thesus Integration
\item LESS support
\item W3C Validation
\item Drag and Drop
\item Auto Prefixer
\item Git Integration for Brackets
\item JS Lint
\end{enumerate}
\subsection{Observations Made}
Here are a list of things that we have found out about Brackets: \newline
\begin{itemize}
\item Brackets is written in JavaScript
\item Its an automated WYSIWYG editor
\item Mainly designed for web developers and front end developers
\item Applies quick edit for HTML elements which helps in displaying corresponding CSS
properties for that particular element.
\item When a color is typed it shows an inline color picker for our better convienience in
searching.
\item While writing the code it enables live preview , that works only for Google Chrome, and if
there is any html syntaxtical error then it forbiddens opening of the live prview.
\item Using this Thesus integration feature that enables inspecting an element in the real time and
debug any extension in brackets.
\item This uses CodeMirror text raw html editor for Code Formatting.
\item Brackets has got more than 20 Dependencies .
\item It consists of a bracket (which is a root module) that pulls in other modules as dependencies.
\item Some files where the required changes are to be made and importing those files so they can
be used in building an editor for OpenEdx platform
\end{itemize}
\subsection{Idea Of Implementation}
In a similar way in which the Open edX platform uses an xmodule(html\_module.py) that
handles the operation of its current html editor, similarly, using the same working principle
we can replace our editor and load the html editor and configure it for proper working.

\section{ContentTools}
\subsection{About}
ContentTools is a JavaScript/CoffeeScript library aimed at building WYSIWYG editors for
HTML content. ContentTools aims to provide both a fully-functional editor that can be used
out of box and a toolkit of classes, a set of tools for performing common editing tasks, and a
history stack for managing undo/redo. Whilst the components provided by the toolkit work
well together, they can also be used or replaced as required. \newline
\subsection{Features}
\begin{enumerate}
\item Easily integrated with any HTML document.
\item Drag and Drop directly within the page.
\item Floating context-sensitive toolbar.
\item Compatible with all major web browsers and operating systems.
\item The javaScript library can transform any HTML page into an WYSIWYG editor.
\item Countless possibilities for building wondrous interactive apps and services.
\item Media Resizing.
\end{enumerate}
\subsection{Observations Made}
\begin{itemize}
\item Allows text content, images, embedded videos, tables and other page content to be
edited, resized, or moved using the Drag and Drop property within the page itself.
\item Requires bower or npm for installation.
\item For building library’s and project, requires grunt node modules and SASS.
\item Uses an HTML string for formatting rather than using an HTML parser written in
JavaScript.
\item Its library uses a minimal finite state machine(FSM) for JavaScript.
\item It consists of an JS library that provides cross-browser support for content selection.
\item It has a javascript library that provides a set of classes for building content editable
HTML elements.
\item The ContentTools editor has already been implemented into the content management
systems of a number of websites.
\item This editor is a recent editor which has got wonderful, simplified and an unique
UserInterface.
\item Its Framework integration includes “Django REST Framework” and “Image
Uploads with Cloudinary”.
\item ContentTools is part of a collection of JavaScript libraries (ContentTools,
ContentEdit, ContentSelect, FSM, HTMLParser) which were developed to aid in the
creation of HTML WYSIWYG editors.
\item It has an ability to configure styles for your content.
\item Contenttools can be easily integrated into the CMS.
\item It is an opensource web-based HTML editor whose working is quite similar to
CodeMirror and TinyMCE, so integrating with the OpenEdx platform would be
quite simpler as all of then belong to the same working methodology.
\end{itemize}
\subsection{Idea Of Implementation}
In a similar way in which the Open edX platform uses an xmodule(html\_module.py) that
handles the operation of its current html editor, similarly, using the same working principle
we can replace our editor and load the html editor and configure it for proper working.  \newline
\newline
Integrating this would be much easier compared to Brackets and Notepad++ as its working
methodology is just as similar to the CodeMirror and TinyMCE which are being used as of
now in the OpenEdx platform. \newline
\subsection{Installation Issues}
Installing Notepad++ and Brackets easy was error free but with ContentTools the only issue
that occurred while installing was related to the npm package installer. \newline
The packages were not completely installed and hence npm threw an error while
installing. \newline \newline
Here is the log file that shows the errors that would be occurring , \newline 
[\url{https://drive.google.com/open?id=1KUe3jcAPg8DpxmhshPUBYPKya_B-oRSt}]
\newline 
\newline
\textbf{SOLUTION :} To resolve this issue here are the things that are to be followed: \newline
\begin{enumerate}
\item Open your terminal and perform the following operations 
\begin{center}\verb=sudo apt-get update synaptic=\end{center}
\item A dailog box opensSearch for npm and node modules
\item Delete those folders manually
\item Save the changes and close the dialog.
\item Open the terminal and type in the following commands
 \begin{center}\verb=npm install --save ContentTools=\end{center}

\end{enumerate}
\section{Advantages Of Approach1 Compared To Approach2}
\begin{enumerate}
\item Upgrading something is always better than integrating something new altogether. As per
our research and observation the versions of CodeMirror and TinyMCE in Open edX are
outdated. Since the latest versions of these two editors fullfill our requirements updating the
editors is much straighforward and sensible than incorporating two completedly new ediitors
into the Open edX platform, since it can lead to a lot of dependency issues.
\item The iframe approach also allows the course content creator to include third party CSS
such as bootstrap, w3css which will make the course even more interactive and boost
student experience. Course creators can style their courses with a variety of such third party
CSS features and make it a lot more interactive for the students.
\item Wrapping data inside the iframes containerizes the data ensuring no mixing of CSS
between two HTML blocks in LMS. This is very important in order to provide an error free
user experience.
\end{enumerate}


\section{References}
The following links would be helpful in learning more about the working of the editors discussed previoulsy
\begin{enumerate}
\item\textbf{Notepad++}
\begin{itemize}
\item \url{https://github.com/notepad-plus-plus/notepad-plus-plus}
\item \url{https://github.com/notepad-plus-plus/notepad-plus-plus/blob/master/scintilla/lexlib/Accessor.cxx}
\item \url{https://github.com/notepad-plus-plus/notepad-plus-plus/blob/master/scintilla/src/AutoComplete.cxx}
\item \url{https://github.com/notepad-plus-plus/notepad-plus-plus/tree/master/scintilla/lexers}
\item \url{https://github.com/notepad-plus-plus/notepad-plus-plus/tree/master/scintilla/lexlib}
\end{itemize}
\item\textbf{Adobe Brackets}
\begin{itemize}
\item \url{https://github.com/adobe/brackets}
\item \url{https://github.com/adobe/brackets/blob/master/src/styles/brackets_codemirror_override.less}
\item \url{https://github.com/adobe/brackets/tree/master/src/LiveDevelopment}
\item \url{https://github.com/adobe/brackets/blob/master/src/LiveDevelopment/Agents/CSSAgent.js}
\item \url{https://github.com/adobe/brackets/blob/master/src/editor/}
\item \url{https://github.com/adobe/brackets/tree/master/src/htmlContent}
\end{itemize}
\item\textbf{ContentTools}
\begin{itemize}
\item \url{https://github.com/GetmeUK/ContentTools}
\item \url{https://github.com/Cotidia/django-contenttools-demo}
\item \url{http://getcontenttools.com/api/content-select}
\item \url{http://getcontenttools.com/tutorials/content-tools-plus-cms}
\end{itemize}
\end{enumerate}


