\chapter{Approach 1 part II intregating with Open edX}
After testing our hybrid local intregation, and setting up our developing environment for Open edX,
the next phase was to integrate it with the running instance of Open edX on our Devstack
\section{Plan of action}
The idea and plan behind intregating our running hybrid instance of CodeMirror in TinyMCE with
Open edX is mentioned below:
\begin{enumerate}
\item\textbf{Indentation and code folding features}
\begin{itemize}
\item Add the latest version of CodeMirror v5.38.0 to the source code of Open edX
leaving the older version untouched.
\item Configure TinyMCE to open the updated version of CodeMirror as its raw HTML
editor.
\item Test the updated CodeMirror in TinyMCE and see if its wroking as expected.
\end{itemize}
\item\textbf{Enabling internal CSS}
\begin{itemize}
\item Understand the working of the XModule html\_module.py which is the widget
which loads the WYSIWYG editor in Open edX.
\item Learn how the html content is getting stored in the MongoDB .
\item The idea was to wrap the html content inside an inframe before it gets stored in the
Database. However this was to be done only when the course authors used internal
CSS in thier courses.
\item Similarly, while loading the content in LMS, it would be wrapped inside and iframe
first and then rendered in order to produce the desired effect.
\end{itemize}
\end{enumerate}

\section{Implementation}
\subsection{Configuring TinyMCE to use updated CodeMirror v5.38.0}
\begin{itemize}
\item The CodeMirror and TinyMCE source files are located in the edx-platfrom can be found on
the following path : \textit{\textbf{edx-platform/common/statis/js/vendor}}
\item Inside the tinymce source code, we placed the updated version of CodeMirror in the plugins
folder for TinyMCE.\newline The path was :
\textit{\textbf{edx-platform/common/static/js/vendor/tinymce/js/tinymce/plugins}}
\item Next we configured TinyMCE to use the updated CodeMirror as its raw HTML editor. This
was done through making changes in the TinyMCE configuration file \textbf{edit.js} located here :\newline
\textbf{\textit{edx-platform/common/lib/xmodule/xmodule/js/src/html/edit.js}}
\item Similarly, while loading the content in LMS, it would be wrapped inside and iframe
first and then rendered in order to produce the desired effect.
\end{itemize}
After performing the following steps we were able to configure TinyMCE to open our updated
instance of CodeMirror v5.38.0 as its raw HTML editor succesfully. Features such as indentation,
code-folding etc were working fine

\subsection{Using Iframes to enable internal CSS}
\subsubsection{Issues faced}
Once we were able to intreagte updated version of CodeMirror with TinyMCE, we tried testing
internal CSS in our new editor. Our observations are mentioned below:
\begin{itemize}
\item  The effect of the internal CSS is only visible inside the WYSIWYG editor i.e inside
TinyMCE.
\item  Once we save our progress the Open edX backend automatically strips the style tag and
removes it. As a result of this, no effect of the CSS is visible in the actual course content.
\end{itemize}

\section{Change in Approach}
\subsection{What happened to the issues}
We tried to look into the issues by looking into openedx backend by logging at several places in
several different xmodules and we failed. We discussed this issue with Aparna Ma’am(Mentor,
IITBombayX team). She said that we were on a right track and we should play with backend
through logs itself. Strangely enough, there were NO logs generated at all!! This lead us to a
roadblock situation where we need to switch to another approach for checking out backend.
However, all of the ways went at least once through logging step which was not working on our
devstack very well. We also discovered that the server provided to us for R\&D
purposes(10.129.27.44) also was unable to generate any logs.
\subsection{Need for change in approach}
Because of these issues, we were facing a major roadblock ahead of us and hence we need to
switch approaches. Now we had 2 more options, Xblock and integrating entirely a new editor into
an already complex openedx system which does not even generate logs ! Another option was to
reinstall devstack and see what causes devstack to not generate logs. But the clock was ticking and
we only hand 2.5 weeks left for the internship period. We could have pursued approach 1 to the end
and maybe end the internship without completing the project but we chose to change approach.
\subsection{Why Xblock approach and not integrating editors such as
contenttools/grapejs ?}
Two main reasons:
\begin{enumerate}
\item If we wanted to integrate any of these with OpenEdx, we would need the logger to be working
perfectly. And as described in the issues above, that is what kept us from pursuing approach 1 itself
\item Xblocks are modular. If we decide to modify the system, we’re forcing the editor changes to all
users. And probably some of these users don’t even need this advanced editor and features. On the
other hand, if we make this into an xblock, it will be used by people who actually need it and know
how to use it. Also one hidden advantage of xblock is that they are largely platform independent.
Most of the xblocks do not need to be updated on every single version of openedx since the core
Xblock API remains unchanged.
\end{enumerate}



